\documentclass[12pt, letterpaper]{article}

\usepackage{amsmath, amsthm, graphicx}

\title{Parallelization of RSVP Dataset Generator}

% Put your name in the author field
\author{Nick Carey}

% this begins the actual text of the document; everything before this is
% refered to as "preamble"
\begin{document}

% the maketitle command formats and inserts the title, author name, and date
\maketitle

\begin{abstract}
Brain Computer Interfaces have the potential to help interactively explore and classify large datasets.
Rapid Serial Visual Presentations, or RSVP, is a BCI method that must be further tested on visual
datasets in order to find novel applications.  We have developed a program that generates abstract visual datasets
on the fly in order to test the capabilities of RSVP.  In order to speed up dataset generation to match the
pace demanded by RSVP, we analyze and implement parallelization of the data generation program.
Using Amdahl's Law, we find the maximum speedup possible.  We compare actual results with the results
predicted by Amdahl's Law and reason about causes for any non-optimal speedup.  
\end{abstract}

\section{Introduction}
Brain Computer Interfaces (BCI) have the potential to become a new tool for the data scientist. 
By reading brain signals rather than mouse and keyboard signals, the loop between computer and human
can be considerably shortened.  Rapid Serial Visual Presentations (RSVP) is one such BCI system that
reads human subject brain signals while rapidly presenting the subject with approximately ten images per second.
The particular RSVP system in question will look for the P300 brain wave.  The P300 brain wave is emitted whenever
the subject sees something interesting.  Therefore, by associating specific images with the P300 wave, images
are classified as 'interesting' or 'non-interesting'.  

The meaning of the P300 wave can be quite vague.  'Interesting' is a very situational adjective.  Therefore it is 
important to investigate which sort of datasets and data visualizations could be classified using a P300 wave.  
In order to better understand the capabilities of RSVP, we have iteratively developed and tested an abstract 
dataset.  Our experimental goal with RSVP is to see if we can tease out a rare, low-dimensional signal 
hidden in a high-dimensional noisy dataset; finding hidden low-dimensional signal is common to some astronomy and 
cyber forensics problems.  

We have had success using RSVP to classify views of the abstract dataset that show some signal versus 
images showing only noise.  However, we 
currently prepare decks of images before running RSVP; in the future we would like to give RSVP online views of 
the abstract 
dataset based on previous RSVP results.  For example, if RSVP triggers on an image showing some of the hidden 
structure, then in the next iteration of RSVP we would like to show similar views of the dataset in order to drill down
completely on the hidden signal.  However, RSVP is very fast and our abstract dataset query program is very slow.  
In order to match the pace demanded by RSVP, our query program must be able to execute and visualize at least ten
queries per second.  

In section 2, we outline the abstract dataset generator and query program and analyze which program sections are 
parallelizable.  Using Amdahl's Law\cite{Amdahl} we calculate optimal speedup.  In section 3, we describe an 
experiment to test our parallel implementation, and in section 4 we show the results of the experiment.  
In section 5 we analyze the results and discuss any causes for non-optimal speedup.

\section{Dataset Generator Program}

The abstract dataset generator program we have developed to test the capabilities of RSVP consists of several steps.
In step one, we create a set of N 3-dimensional coordinates that form a wire-frame cube.  We then add on a few more 
dimensions of uniform random noise.  After step one, we have a starting dataset of six dimensions, three of which form 
coordinates for a wire-frame cube and three of which are random noise. 

In step two, we apply a random six-dimensional rotation to our starting dataset.  After the random rotation, we have 
our simulated dataset.  Visualizations of the simulated dataset should only show noise; we have hidden our 
low-dimensional cube signal inside a high dimensional noisy dataset.  The simulated dataset is what models 
some astronomy and cyber datasets. 

In step three, we query the simulated dataset 1500 times, as needed by an iteration of the RSVP system.  
Each query is another random six-dimensional rotation applied
to the simulated dataset.  Some rare random rotations will be 'lucky' and close to the opposite of the original 
six-dimensional rotation used to create the simulated dataset; visualizations of these 'lucky' rotations
will show some of the hidden cubic signal structure while most random rotations will only show noise.

In step four, we visualize the random rotations applied to the simulated dataset to create images 
appropriate for use with RSVP.

\subsection{Amdahl's Law Analysis}





\section{Experimental Method and Setup}



\section{Results}

\section{Discussion}


\bibliographystyle{plain}

\bibliography{templateBibliography}

\end{document}
